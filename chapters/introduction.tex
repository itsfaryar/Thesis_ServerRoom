
\section{Backgrounds}
Server rooms are a critical component of today's technology-driven world, with the growing reliance on and need for technological devices, ensuring that server rooms function properly has become an essential part of our daily lives. Server rooms are important to an organization because they contain infrastructure and critical equipment. Monitoring various parameters such as temperature, humidity, electricity and others helps ensure that the system is running smoothly and the first step in monitoring server rooms is to understand the different parameters that need to be monitored. 
\subsection{Monitoring the environment}
    A monitoring system should be able to monitor the environment of the server room and alert the administrators in case of any abnormality. The following list contains some of the most important parameters that should be monitored in a server room.
    \begin{itemize}
        \item Temperature
            \begin{description}
                \item Server room temperature should be maintained at a stable, suitable level to ensure the proper functioning of equipment. The recommended range of temperatures is between 18°C and 27°C.  \cite{ASHRAE_Storage_White_Paper_2015}
                \begin{description}
                    \item[Too high temperature:] If the temperature inside a server room rises above the recommended range, it can cause several problems. High temperatures can make equipment operate less efficiently and potentially fail altogether. High temperatures can also decrease the life expectancy of electronic devices and increase the chance that stored data will be corrupted. \cite{data_center_cooling}
                    \item[Too low temperature:] If the temperature in a server room is allowed to fall below the recommended range, it can cause several issues. The cold air can lead to condensation—which leads directly to corrosion and equipment damage. In addition, low temperatures can decrease the efficiency of equipment and make it more likely to fail. \cite{data_center_cooling}
                \end{description}
            \end{description}
        \item Humidity
            \begin{description}
                \item On the other hand, High humidity levels can lead to condensation, which causes corrosion of the equipment and short circuits. High humidity levels can also create conditions that are conducive to the growth of mold and other microorganisms, which in turn damage equipment and affect indoor air quality. \cite{10.1115/IPACK2015-48176}
            \end{description}
        \item Dust
            \begin{description}
                \item Dust accumulation in a server room can degrade the performance and lifespan of equipment. Dust can block air vents, causing overheating, and it also attracts moisture leading to corrosion or other problems. So monitoring the levels of dust in a server room can help identifying and addressing potential problems.
                    \begin{description}
                        \item The suitable range for dust density in a server room is around 0 to 3mg/m3.
                    \end{description}
            \end{description}
        \item Water Leakage
            \begin{description}
                \item Serious consequences can result from water leakage in a server room, even if only a small amount of water leaks onto equipment. It is important to have proper monitoring systems and alerts that will notify personnel as soon as possible after any leak occurs. \cite{telecommunications2010tia}
            \end{description}
        \item Electricity
            \begin{description}
                \item Monitoring the state of electricity, including voltage and current, is important in a server room to ensure the stability and reliability of the power supply to the equipment. Electrical voltage and current fluctuations can lead to problems with electronic equipment, such as data loss and corruption. In order to minimize these risks, server rooms are typically equipped with uninterruptible power supplies (UPS) and surge protection devices that help stabilize the voltage. The following list contains some common ranges for voltage and current.\cite{telecommunications2010tia}
                \begin{description}
                    \item[Voltage:] The recommended operating range in from 208V to 240V and the maximum recommended limit is 264V.
                    \item[Current:] The recommended operating range in from 20A to 40A (per phase)
                \end{description}
            \end{description}
        \item Movement
            \begin{description}
                \item Movement sensors, also known as motion detectors, can be used in order to detect unauthorized access to the room by detecting movements within the room and alerting the administrators.
            \end{description}
        \item Smoke
            \begin{description}
                \item Smoke sensors can be used to detect smoke and alert the administrators in case of a fire.
            \end{description}
        \item Authorization
            \begin{description}
                \item Authorization devices, such as cameras, fingerprints, keypads, etc., can be used to detect unauthorized access to the room by detecting the presence of a specific person.
            \end{description}
        \end{itemize}
        \subsection{Security}
        Another aspect of a monitoring system is that it should be able to prevent any intrusions and ensure that data remains intact and that access is restricted to only those with the appropriate privileges.
        \begin{itemize}
            \item Confidentiality And Authentication
                \begin{description}
                    \item in every IOT system, a proper authentication is crucial to ensure the security of the data so it wouldn't get into the wrong hands or get mutated by malicious actors and because the data is being transmitted remotely, more security measures has to be taken. \cite{saba2022anomaly,stallings2007network}
                \end{description}
            \item Authorization
            \begin{description}
                \item Security and reliability are the most important factors in our use case of network and other security measure implementations. There are trade offs like losing flexibility but these are the trade offs that we are willing to make to reach the maximum level of security possible.\cite{saba2022anomaly,stallings2007network}
            \end{description}
        \item Integrity
            \begin{description}
                \item In an IOT system, there is a risk of intruders attempting to impersonate legitimate users and alter or manipulate data. So it is important to implement verification and encryption methods to make sure in an event of a breach, the integrity of the data is still ensured and no nefarious actor in the system can mutate or access the data in any way because the consequences can be quite catastrophic. \cite{saba2022anomaly,stallings2007network}
            \end{description}
        \end{itemize}
        \subsection{Real-time monitoring}
        Real-time monitoring is important in a monitoring system to ensure that the administrators are notified as soon as possible in case of any abnormality.
        \subsection{Data Visualization}
        Data visualization is an important part of a monitoring system, because it can provide a clear and compact representation of the data sent by the sensors. This allows administrators to quickly and easily identify trends, patterns, and anomalies in the server room environment, such as changes in temperature, humidity, power, and other environmental factors.\\
        Data visualization can be achieved through a variety of means, such as graphs, charts, and maps. These visual representations can provide a lot of information in real-time, allowing administrators to make informed decisions about the state of their server room and respond quickly to any issues or problems. For example, graphs can be used to track trends over time.

        \section{Purpose}
            The purpose of this thesis is to demonstrate how an effective and efficient monitoring system can be implemented for server rooms using sensors, which are introduced in three categories in the tables \ref{table:electrical_sensors}, \ref{table:envirement_sen} and \ref{table:security_sen}. The system will be designed to provide real-time data to a website, so that it can be monitored anywhere with internet access. This thesis will examine how this system was designed and implemented, including the selection of sensors and development of an online panel for monitoring data visualization.
            
            \begin{table}
                \centering
                \caption{Electrical sensors}
                \begin{tabular}{ |c|c|c|c|c|c|}
                \hline
                {\textbf{Name}} & {\textbf{Measurement Unit}} & {\textbf{Output}} \\ 
                \hline

                Voltage Sensor & Voltage &  ADC \\
                \hline
                
                Current Sensor & Amper &  ADC \\
                \hline
                \end{tabular}
                \label {table:electrical_sensors}
            \end{table}
               
            \begin{table}
                \centering
                \caption{Environment sensors}
                \begin{tabular}{ |c|c|c|c|c|c|}
                    \hline
                    {\textbf{Name}} & {\textbf{Measurement Unit}} & {\textbf{Operating temp}}& {\textbf{Desired values}}   \\ 
                    \hline

                    Temperature sensor & Celsius & -20°C - 50°C & 18°C - 27°C\\
                    \hline
                    Smoke sensor & ADC & -20°C - 50°C & -  \\
                    \hline
                    Humidity sensor & Percent &  -20°C - 50°C & 40\% - 60\% \\
                    \hline
                    Water Leakage Sensor & - & -20°C - 50°C & -  \\
                    \hline
                    Dust sensor &  mg/m3 & -20°C - 50°C & 0 - 3  \\
                    \hline
                \end{tabular}
                \label {table:envirement_sen}
            \end{table}
        
            \begin{table}
                \centering
                \caption{Security sensors}
                \begin{tabular}{ |c|c|c|c|c|c|}
                    \hline
                    {\textbf{Name}}  \\ 
                    \hline

                    Movement sensor  \\
                    \hline
                    Fingerprint sensor  \\
                    \hline
                    Camera \\
                    \hline
                \end{tabular}
                \label{table:security_sen}
            \end{table}
          


